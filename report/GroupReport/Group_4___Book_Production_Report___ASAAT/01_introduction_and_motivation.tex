%Introduction and motivate the problem

Modern production systems are becoming increasingly more complex, requiring coordination of multiple independent components that must operate efficiently, flexibly, and reliably, which represent the shift to Industry 4.0 environments \cite{ibm_industry4}. Traditional linear production models, where tasks are performed strictly in sequence, are often insufficient for such environments. Instead, many new and real-world systems require parallel processing, dynamic task allocation, and the ability to adapt to changes, new features, or failures during operation. 

Book production serves as an example of these challenges. Producing a book involves several distinct steps such as printing pages, printing covers, binding, and packaging that must be completed correctly to deliver a finished product. While some of these can be executed independently or in parallel, others depend on the completion of the preceding tasks. coordinating these processes efficiently is essential to minimize waiting times, balance workloads across production cells, and ensure a timely completion of the product. 

As the production system grows in scale and complexity, manual coordination or rigid workflows become impractical. There is a clear need for a software-based solution capable of managing task distribution, handling both linear and non-linear dependencies between production tasks, and to adapt to changes in the production process. Furthermore, the system must be robust enough to handle failures in individual production units and flexible enough to integrate new production steps as requirements evolve. 

This article focuses on addressing these challenges by exploring the design of a scheduling system for book production. The goal is to define a system that can effectively distribute work/tasks among multiple production cells, enforce the correct execution order, support parallel processing, and remain extensible and fault tolerant. The following chapter formalizes this problem and introduces the key questions that guide the proposed solution. 
